% !TEX encoding = IsoLatin
%% Antes de processar este arquivo LaTeX (LaTeX2e) deve
%% verificar que o arquivo TEMA.cls esta na mesma
%% pasta. O arquivo TEMA.cls pode ser obtido do
%% endereco http://tema.sbmac.org.br.

\documentclass{TEMA}

\usepackage[brazil]{babel}      % para texto em Português
%\usepackage[english]{babel}    % para texto em Inglês
 
\usepackage[latin1]{inputenc}   % para acentuação em Português IsoLatin1
%\usepackage[utf8]{inputenc}     % para acentuação em Português UTF-8

% Atenção: O presente arquivo usa a codificação IsoLatin1. Para
% utilizar a codificação UTF-8, converter o arquivo usando seu
% programa favorito, então troque o argumento do inputenc 
% adequadamente.

\usepackage[dvips]{graphics}
\usepackage{subfigure}
\usepackage{graphicx}
\usepackage{epsfig}
\usepackage{hyperref}
\usepackage{framed}
\usepackage{psfrag}
\usepackage{tikz} 

\newcommand{\B}{{\tt\symbol{92}}}
\newcommand{\til}{{\tt\symbol{126}}}
\newcommand{\chap}{{\tt\symbol{94}}}
\newcommand{\agud}{{\tt\symbol{13}}}
\newcommand{\crav}{{\tt\symbol{18}}}

\begin{document}

%********************************************************
\title{
     Modelo com Instruções para Preparação \\ 
     de Trabalhos para TEMA%
     \thanks{Agradecimentos por auxílio. Baseado no documento
     proposto por A. Sri Ranga e E. Wendland.}
}

\author{
     F. S. SOUSA%
     \thanks{fsimeoni@icmc.usp.br; Editor executivo, desde 2014.},
     Departamento de Matemática Aplicada e Estatística,
     ICMC, Universidade de São Paulo, 
     Av. Trabalhador São-carlense, 400, 
     13566-590, São Carlos, SP, Brasil.
     \\ \\
     A. R. L. OLIVEIRA%
     \thanks{aurelio@ime.unicamp.br; Editor chefe, desde 2015},
     Departamento de Matemática Aplicada,
     IMECC, Universidade Estadual de Campinas,
     R. Sérgio Buarque de Holanda, 651, 
     13083-859, Campinas, SP, Brasil. 
}

\criartitulo

\runningheads{F.S. Sousa e A.R.L. Oliveira}%
{Instruções para Preparação de Trabalhos}

\begin{abstract}

{\bf Resumo}. Este documento, preparado usando-se a classe
especial \texttt{TEMA.cls}, fornece algumas informações 
importantes para os autores que pretendem submeter 
trabalhos (artigos) completos para a série TEMA.

{\bf Palavras-chave}. Palavra-chave 1, palavra-chave 2,
palavra-chave 3.

\end{abstract}


%********************************************************
\section{Introdução}

    A série TEMA - Tendências em Matemática
Aplicada e Computacional, com título em inglês ``Trends in Applied
and Computational Mathematics'',  tem como objetivo principal
publicar traba\-lhos completos originais, de no máximo 20 páginas,
de todas as áreas de Matemática Aplicada e Computacional.
Excepcionalmente, a critério do Comitê Editorial, poderão ser
publicados trabalhos com mais de 20 páginas.  Um volume, composto
de até três números, é publicado anualmente. A finalidade 
da série é servir como veículo para publicação de artigos originais 
sobre temas de interesse dos associados da SBMAC, exclusivamente
através de submissão na modalidade fluxo contínuo. Autores de
trabalhos submetidos e publicados nos anais do CNMAC são encorajados
a estenderem seus trabalhos e submeterem uma versão completa 
e atualizada à TEMA, que passará pela mesma análise dos 
artigos submetidos via fluxo contínuo. 

\subsection{Fluxo contínuo da série TEMA}

Os autores que manifestarem interesse em submeter {\bf trabalhos completos}
à publicação na série TEMA em qualquer época do ano, deverão
prepará-los em \LaTeX2e. A versão {\bf .tex} e a
versão {\bf .pdf} de cada trabalho a ser considerado para
publicação, juntamente com os arquivos de figuras (se houver),
deverão ser submetidas de forma eletrônica no endereço \url{http://tema.sbmac.org.br}.

Maiores detalhes sobre o calendário de publicações da TEMA podem ser encontrados em \url{http://tema.sbmac.org.br}.

Os trabalhos submetidos serão avaliados por consultores {\em ad
hoc} e, os selecionados, serão publicados no próximo número da
TEMA, de acordo com a demanda e com o calendário de publicações.

Devido a grande demanda para publicação, o número máximo de
páginas para trabalhos está fixado em {\bf 20}.\\

Os autores, no lugar de \verb!\documentclass{article}!, 
deverão usar o comando \verb!\documentclass{TEMA}!, o 
``class file'' \texttt{TEMA.cls} deve estar no mesmo 
diretório no momento da compilação e pode ser
obtido no mesmo endereço eletrônico  \newline 
\url{http://tema.sbmac.org.br}.

O ``class file'' \texttt{TEMA.cls} foi criado para que todos os
trabalhos enviados para publicação em TEMA sejam padronizados.
Assim, todos os trabalhos terão tamanho de fonte {\bf 10pt} e área
de impressão: {\bf 19.0cm} por {\bf 12.7cm}.

Por motivo de padronização, a linha seguinte a 
\verb!\documentclass{TEMA}! deve ser 
\verb!\usepackage[brazil]{babel}! para os trabalhos escritos em
português e \newline
\verb!\usepackage[english]{babel}! para os trabalhos escritos em inglês.

%********************************************************

\section{Página inicial do trabalho}

A primeira página do trabalho deve conter o título do
trabalho, nomes e endereços dos autores e resumo para
trabalhos escritos em português (abstract para os
escritos em inglês). Nos trabalhos escritos em
português (inglês) deve ser incluído, ao final do texto, antes das
referências, um {\bf abstract} em inglês ({\bf resumo} em português).
Como exemplo, ilustramos como construir a primeira página
deste documento.

Após o comando \verb!\begin{document}! inserir as seguintes
instruções:
\begin{framed}
\begin{verbatim}
\title{
     Modelo com Instruções para Preparação \\ 
     de Trabalhos para TEMA%
     \thanks{Agradecimentos por auxílio. Baseado no documento
     proposto por A. Sri Ranga e E. Wendland.}
}

\author{
     F. S. SOUSA%
     \thanks{fsimeoni@icmc.usp.br; Editor executivo, desde 2014.},
     Departamento de Matemática Aplicada e Estatística,
     ICMC, Universidade de São Paulo, 
     Av. Trabalhador São-carlense, 400, 
     13566-590, São Carlos, SP, Brasil.
     \\ \\
     A. R. L. OLIVEIRA%
     \thanks{aurelio@ime.unicamp.br; Editor chefe, desde 2015},
     Departamento de Matemática Aplicada,
     IMECC, Universidade Estadual de Campinas,
     R. Sérgio Buarque de Holanda, 651, 
     13083-859, Campinas, SP, Brasil. 
}

\criartitulo

\runningheads{F.S. Sousa e A.R.L. Oliveira}%
{Instruções para Preparação de Trabalhos}

\begin{abstract}

{\bf Resumo}. Este documento, preparado usando-se a classe
especial \texttt{TEMA.cls}, fornece algumas informações 
importantes para os autores que pretendem submeter 
trabalhos (artigos) completos para a série TEMA.

{\bf Palavras-chave}. Palavra-chave 1, palavra-chave 2,
palavra-chave 3.

\end{abstract}
\end{verbatim}
\end{framed}


Nos trabalhos em inglês, substituir \verb!{\bf Resumo}! por
\verb!{\bf Abstract}! e substituir \verb!{\bf Palavras-chave}!
por \verb!{\bf Keywords}!.

Os parâmetros de \verb!\runningheads{ }{ }! são as
informações impressas nos cabeçalhos (``headings'') das páginas
pares e ímpares, respectivamente.  O primeiro parâmetro refere-se
aos sobrenomes dos autores e o segundo, ao título abreviado do
trabalho (máximo de 70\% da largura do texto). Quando há vários
autores, isto é, quando os sobrenomes dos autores ocupam mais que
50\% da largura do texto, o primeiro parâmetro deve ser o
sobrenome do primeiro autor seguido de ``\textbf{et al.}''.

Quando há mais de um autor, seguir o seguinte exemplo:  

\newpage

\begin{framed}
\begin{verbatim}
\author{
     F. S. SOUSA%
     \thanks{fsimeoni@icmc.usp.br; Editor executivo, desde 2014.},
     A. CASTELO%
     \thanks{castelo@icmc.usp.br; Editor chefe, 2013-2015.}
     Departamento de Matemática Aplicada e Estatística,
     ICMC, Universidade de São Paulo, 
     Av. Trabalhador São-carlense, 400, 
     13566-590, São Carlos, SP, Brasil.
     \\ \\
     A. R. L. OLIVEIRA%
     \thanks{aurelio@ime.unicamp.br; Editor chefe, desde 2015},
     Departamento de Matemática Aplicada,
     IMECC, Universidade Estadual de Campinas,
     R. Sérgio Buarque de Holanda, 651, 
     13083-859, Campinas, SP, Brasil. 
}
\end{verbatim}
\end{framed}

Neste caso, a instrução \verb!\runningheads{}{}! poderá ficar como: 
\begin{framed}
\begin{verbatim}
\runningheads{Sousa et al.}{Instruções para Autores}
\end{verbatim}
\end{framed}
\noindent ou
\begin{framed}
\begin{verbatim}
\runningheads{Sousa, Castelo e Oliveira}{Instruções para Autores}
\end{verbatim}
\end{framed}


%********************************************************

\section{Sobre equações}

Embora se deva usar o comando \verb!\documentclass{TEMA}!, as
equações e referências bibliográficas são geradas da mesma forma
quando se a classe ``article'' com o comando  \verb!\documentclass{article}!.

A versão mais atual do arquivo \texttt{TEMA.cls} elimina a necessidade
de utilizar o comando \verb!\newsec{Nome da Seção}!, como se fazia
anteriormente. Agora os autores podem utilizar o comando
\verb!\section{Nome da seção}! como se faz na classe ``article'',
sem qualquer modificador extra.
O comando \verb!\newsec{Nome da Seção}! será mantido apenas por
questões de compatibilidade com textos gerados em versões anteriores.
O seguinte exemplo mostra o resultado final do uso desses comandos: 

\begin{framed}
\begin{minipage}{10cm}
\setcounter{section}{0}
\begin{verbatim}
\section{Primeira seção}
\label{cin} Considere
\begin{equation} \label{cin.um}
   \begin{array}{rcl}
     S_{n+1}(z) & = & z S_{n}(z) +
     a_{n+1} S_{n}^{*}(z), \\ [1ex]
     \left(1-|a_{n+1}|^2\right) z S_{n}(z) & = &
     S_{n+1}(z) - a_{n+1} S_{n+1}^{*}(z),
   \end{array} 
\end{equation}
para $n \geq 1$, onde $S_n^{*}(z) = z^n \overline{S}_n(1/z)$.
As equações (\ref{cin.um}) acima são as primeiras
equações numeradas desta seção. Abaixo, um exemplo de  equação
centralizada mas não numerada.
\[ x^x = e^{x \ln(x)}, \qquad x > 0. \]

\section{Segunda seção}
\label{qua}

A equação (\ref{qua.um}) é a primeira equação numerada da seção
\ref{qua}, veja
\begin{equation} \label{qua.um}
e = \lim_{n\rightarrow\infty} \left( 1+\frac{1}{n}\right)^n.
\end{equation}

\subsection{Primeira subseção da segunda seção} 
Observe que as equações continuam sendo numeradas de acordo
com a seção.
\begin{eqnarray} \label{qua.dois}
A_{j} &=&\sum_{k=0}^j a_k +\sum_{k=j+1}^{\infty} b_k c_k, \\
B_{j} &=&\sum_{k=0}^j b_k +\sum_{k=j+1}^{\infty} a_k c_k,\\
T_{j} &=&\prod_{k=0}^j a_k+\prod_{k=j+1}^{2j} b_k c_k.\nonumber
\end{eqnarray}

% compatibilidade com o antigo comando \newsec
\newsec{Terceira Seção} 

Na seção \ref{qua} vimos....
\end{verbatim}
\end{minipage}
\end{framed}

\begin{framed}
\begin{minipage}{10cm}
\setcounter{section}{0}

\section{Primeira seção}
\label{cin} Considere
\begin{equation} \label{cin.um}
   \begin{array}{rcl}
     S_{n+1}(z) & = & z S_{n}(z) +
     a_{n+1} S_{n}^{*}(z), \\ [1ex]
     \left(1-|a_{n+1}|^2\right) z S_{n}(z) & = &
     S_{n+1}(z) - a_{n+1} S_{n+1}^{*}(z),
   \end{array} 
\end{equation}
para $n \geq 1$, onde $S_n^{*}(z) = z^n \overline{S}_n(1/z)$.
As equações (\ref{cin.um}) acima são as primeiras
equações numeradas desta seção. Abaixo, um exemplo de  equação
centralizada mas não numerada.
\[ x^x = e^{x \ln(x)}, \qquad x > 0. \]

\section{Segunda seção}
\label{qua}

A equação (\ref{qua.um}) é a primeira equação numerada da seção
\ref{qua}, veja
\begin{equation} \label{qua.um}
e = \lim_{n\rightarrow\infty} \left( 1+\frac{1}{n}\right)^n.
\end{equation}

\subsection{Primeira subseção da segunda seção} 
Observe que as equações continuam sendo numeradas de acordo
com a seção.
\begin{eqnarray} \label{qua.dois}
A_{j} &=&\sum_{k=0}^j a_k +\sum_{k=j+1}^{\infty} b_k c_k, \\
B_{j} &=&\sum_{k=0}^j b_k +\sum_{k=j+1}^{\infty} a_k c_k,\\
T_{j} &=&\prod_{k=0}^j a_k+\prod_{k=j+1}^{2j} b_k c_k.\nonumber
\end{eqnarray}

% compatibilidade com o antigo comando \newsec
\newsec{Terceira Seção} 

Na seção \ref{qua} vimos....

\end{minipage}
\end{framed}


%%%%%%%%%%%%%%%%%%%%%%%%%%%%%%%%
\newpage
\section{Sobre figuras e tabelas}

As figuras e ilustrações {\bf não} podem ser coloridas e, de
preferência, devem ser preparadas em formato ``Portable 
Document Format'' (\texttt{.pdf}), ``Encapsulated
Postscript'' (\texttt{.eps}) ou ``postscript'' (\texttt{.ps}).

Anotações e símbolos nas figuras devem ser visíveis e compatíveis
com o tamanho padrão de fonte do manuscrito. Pode-se utilizar 
pacotes como \texttt{psfrag} ou gerar figuras no formato
\texttt{.pdf\_t}. Exemplos da utilização destes pacotes 
estão listados a seguir.

\subsection{Comando \texttt{psfrag}}

\noindent\textbf{Atenção:} O comando \texttt{psfrag} funciona
apenas quando se compila o arquivo \texttt{.tex} com os comandos
\texttt{latex} para gerar o arquivo \texttt{.dvi}, antes de 
transformá-lo posteriormente em \texttt{.ps} ou \texttt{.pdf}
com o programa de sua preferência. Se a compilação for feita 
diretamente com \texttt{pdftex} ou \texttt{pdflatex}, o 
comando \texttt{psfrag} não conseguirá trocar as macros da 
imagem original por texto \LaTeX. Exemplo de resutado do 
código abaixo está ilustrado na figura \ref{fig:01}.

\begin{framed}
\begin{verbatim}
\begin{figure}[h!]
  \psfrag{_err}[c][c][1][180]{$\|p-\mathcal Ip\|_{L^2(\Omega)}$}
  \psfrag{_h}[c][c]{$h$}
  \psfrag{_h3meios}[l][l]{$h^\frac{3}{2}$}
  \psfrag{_hmeio}[l][l]{$h^\frac{1}{2}$}
  \psfrag{_mini}[l][c]{$Q_h^1$}
  \psfrag{_exp}[l][c]{$Q_h^\Gamma$}
  \centering
  \includegraphics*[width=0.7\linewidth]{fig01.eps}
  \caption{Exemplo de imagem \texttt{eps} com anotação 
           utilizando o pacote \texttt{psfrag}. Imagem
           retirada de \cite{Ausas:2010}.}
  \label{fig:01}
\end{figure}
\end{verbatim}
\end{framed}


\begin{figure}[h!]
  \psfrag{_err}[c][c][1][180]{$\|p-\mathcal Ip\|_{L^2(\Omega)}$}
  \psfrag{_h}[c][c]{$h$}
  \psfrag{_h3meios}[l][l]{$h^\frac{3}{2}$}
  \psfrag{_hmeio}[l][l]{$h^\frac{1}{2}$}
  \psfrag{_mini}[l][c]{$Q_h^1$}
  \psfrag{_exp}[l][c]{$Q_h^\Gamma$}
  \centering
  \includegraphics*[width=0.7\linewidth]{fig01.eps}
  \caption{Exemplo de imagem \texttt{eps} com anotação 
           utilizando o pacote \texttt{psfrag}. Imagem
           retirada de \cite{Ausas:2010}.}
  \label{fig:01}
\end{figure}



\subsection{Arquivo \texttt{.pdf\_t}}

Nos arquivos com formato \texttt{.pdf\_t}, todo o texto é automaticamente
trocado por textos \LaTeX, inclusive fórmulas e anotações. Isso é feito
através de um arquivo auxiliar ao \texttt{.pdf} com extensão 
\texttt{.pdf\_t}, que deve ser importado via comando \texttt{input}.
Os arquivos deste tipo são gerados por programas de desenho, sendo 
o \texttt{inkscape} o mais popular (multiplataforma, grátis).
O resultado do seguinte código está ilustrado na figura \ref{fig:02}.


\begin{framed}
\begin{verbatim}
\begin{figure}[h!]
  \centering
  \scalebox{0.24}{\input{fig02.pdf_t}}
  \caption{Exemplo da utilização de figura no formato 
           \texttt{pdf\_t}. Retirada de \cite{Ausas:2010}.} 
  \label{fig:02}
\end{figure}
\end{verbatim}
\end{framed}

\begin{figure}[h!]
  \centering
  \scalebox{0.24}{\input{fig02.pdf_t}}
  \caption{Exemplo da utilização de figura no formato 
           \texttt{pdf\_t}. Retirada de \cite{Ausas:2010}.} 
  \label{fig:02}
\end{figure}

\subsection{Pacote \texttt{tikz}}

O \texttt{tikz} e uma poderosa ferramenta para desenho direto
em \LaTeX, gerando figuras vetorias de altíssima qualidade. O
usuário deve no entanto se familiarizar com a geração do script 
que resultará no desenho. Sua utilização é aceita no modelo
da revista TEMA. O código a seguir combina a utilização do 
pacote com \texttt{minipage} e \texttt{subfigure}, e o 
resultado está ilustrado na figura \ref{fig:03}.


\begin{framed}
\begin{verbatim}
\begin{figure}[h!]
\centering
	\begin{minipage}[b]{0.45\linewidth}
	\centering
		\subfigure[\emph{Wall}-Norte]{
			\centering
			\begin{tikzpicture}[scale=1.5]
				\draw[-] (1,1) rectangle (3,3);
				\draw[dashed, lightgray] (1,2) -- (3,2);
				\draw[dashed, lightgray] (2,1) -- (2,3);
				\draw[fill = black] (2,2) circle (0.06cm) 
				     node[below, scale=0.8] {$p_{i,j}$};
				\node[scale=0.8] at (2,3.2) {$\mathbf{u} = 0$};
				\draw[fill = black] (2,1) circle (0.06cm) 
				     node[below right, scale=0.8] {$v_{i,j-1/2}$};
				\draw[fill = black] (3,2) circle (0.06cm) 
				     node[right, scale=0.8] {$u_{i+1/2,j}$};
				\draw[dashed, gray] (3,3) -- (3,4);
				\draw[fill = gray, color=gray] (3,4) circle (0.06cm)
				     node[right, scale=0.8, gray] {$u_{i+1/2,j+1}$};
			\end{tikzpicture}
		\label{norte} }
	\end{minipage}
	\begin{minipage}[b]{0.45\linewidth}
	\centering
		\subfigure[\emph{Wall}-Sul]{
			\centering
			\begin{tikzpicture}[scale=1.5]
				\draw[-] (1,1) rectangle (3,3);
				\draw[dashed, lightgray] (1,2) -- (3,2);
				\draw[dashed, lightgray] (2,1) -- (2,3);
				\draw[fill = black] (2,2) circle (0.06cm) 
				     node[below, scale=0.8] {$p_{i,j}$};
				\node[scale=0.8] at (2,0.8) {$\mathbf{u} = 0$};
				\draw[fill = black] (2,3) circle (0.06cm) 
				     node[above right, scale=0.8] {$v_{i,j+1/2}$};
				\draw[fill = black] (3,2) circle (0.06cm) 
				     node[right, scale=0.8] {$u_{i+1/2,j}$};
				\draw[dashed, gray] (3,1) -- (3,0);
				\draw[fill = gray, color=gray] (3,0) circle (0.06cm) 
				     node[right, scale=0.8, gray] {$u_{i+1/2,j-1}$};
			\end{tikzpicture}
			\label{sul} }
	\end{minipage}
	\caption{Exemplo da utilização do pacote \texttt{tikz}.}
	\label{fig:03}
\end{figure}
\end{verbatim}
\end{framed}


\begin{figure}[h!]
\centering
	\begin{minipage}[b]{0.45\linewidth}
	\centering
		\subfigure[\emph{Wall}-Norte]{
			\centering
			\begin{tikzpicture}[scale=1.5]
				\draw[-] (1,1) rectangle (3,3);
				\draw[dashed, lightgray] (1,2) -- (3,2);
				\draw[dashed, lightgray] (2,1) -- (2,3);
				\draw[fill = black] (2,2) circle (0.06cm) 
				     node[below, scale=0.8] {$p_{i,j}$};
				\node[scale=0.8] at (2,3.2) {$\mathbf{u} = 0$};
				\draw[fill = black] (2,1) circle (0.06cm) 
				     node[below right, scale=0.8] {$v_{i,j-1/2}$};
				\draw[fill = black] (3,2) circle (0.06cm) 
				     node[right, scale=0.8] {$u_{i+1/2,j}$};
				\draw[dashed, gray] (3,3) -- (3,4);
				\draw[fill = gray, color=gray] (3,4) circle (0.06cm)
				     node[right, scale=0.8, gray] {$u_{i+1/2,j+1}$};
			\end{tikzpicture}
		\label{norte} }
	\end{minipage}
	\begin{minipage}[b]{0.45\linewidth}
	\centering
		\subfigure[\emph{Wall}-Sul]{
			\centering
			\begin{tikzpicture}[scale=1.5]
				\draw[-] (1,1) rectangle (3,3);
				\draw[dashed, lightgray] (1,2) -- (3,2);
				\draw[dashed, lightgray] (2,1) -- (2,3);
				\draw[fill = black] (2,2) circle (0.06cm) 
				     node[below, scale=0.8] {$p_{i,j}$};
				\node[scale=0.8] at (2,0.8) {$\mathbf{u} = 0$};
				\draw[fill = black] (2,3) circle (0.06cm) 
				     node[above right, scale=0.8] {$v_{i,j+1/2}$};
				\draw[fill = black] (3,2) circle (0.06cm) 
				     node[right, scale=0.8] {$u_{i+1/2,j}$};
				\draw[dashed, gray] (3,1) -- (3,0);
				\draw[fill = gray, color=gray] (3,0) circle (0.06cm) 
				     node[right, scale=0.8, gray] {$u_{i+1/2,j-1}$};
			\end{tikzpicture}
			\label{sul} }
	\end{minipage}
	\caption{Exemplo da utilização do pacote \texttt{tikz}.}
	\label{fig:03}
\end{figure}

\subsection{Outros comandos}

Outros comandos considerados padrão em \LaTeX também podem ser
utilizados, como o pacote \texttt{pstool} e o comando \texttt{epsfig}.
Figuras nos formatos \texttt{.png} e \texttt{.jpg} são desencorajadas,
mas podem ser utilizadas, desde que em ótima resolução e máxima
qualidade (sem compressão).

\subsection{Tabelas}

Tabelas seguem o padrão \LaTeX. A tabela \ref{tabela:01}  é um exemplo simples.

\begin{framed}
\begin{verbatim}
\begin{table} [h]
\caption{Exemplo de tabela.} \label{tabela:01}
\begin{center}
\begin{tabular}{|c|c|c|c|c|}
\hline  & A & B & C & D \\
\hline
\hline 0 & \multicolumn{1}{r|}{1.00}   & \multicolumn{1}{l|}{2.0}   
         & \multicolumn{2}{c|}{7.0} \\
\hline 1 & \multicolumn{1}{r|}{1.00}  & \multicolumn{1}{l|}{2.00}  
         & 3.000  & 4.000  \\
\hline 2 & \multicolumn{1}{r|}{1.000} & \multicolumn{1}{l|}{2.000} 
         & 3.00   & 4.0000   \\
\hline
\end{tabular}
\end{center}
\end{table}
\end{verbatim}
\end{framed}

\begin{table} [h]
\caption{Exemplo de tabela.} \label{tabela:01}
\begin{center}
\begin{tabular}{|c|c|c|c|c|}
\hline  & A & B & C & D \\
\hline
\hline 0 & \multicolumn{1}{r|}{1.00}   & \multicolumn{1}{l|}{2.0}   
         & \multicolumn{2}{c|}{7.0} \\
\hline 1 & \multicolumn{1}{r|}{1.00}  & \multicolumn{1}{l|}{2.00}  
         & 3.000  & 4.000  \\
\hline 2 & \multicolumn{1}{r|}{1.000} & \multicolumn{1}{l|}{2.000} 
         & 3.00   & 4.0000   \\
\hline
\end{tabular}
\end{center}
\end{table}


%%%%%%%%%%%%%%%%%%%%%%%%%%%%%%%%%%%%%%%
\section{Algumas padronizações}

\begin{itemize}

\item Numerar apenas as equações a serem referenciadas;

\item Para centralizar ou destacar equações, utilizar os comandos: 
\verb!\begin{equation*}! e \verb!\end{equation*}!,
ou simplesmente,
\verb!\[! e \verb!\]!. 
Estes comandos destacam as equações sem numerá-las.

\item Para fazer referência a uma equação, utilizar a combinação 
de comandos \verb!\label{}! e \verb!\ref{}! da seguinte forma: \\
\begin{framed}
\begin{verbatim}
\begin{equation} \label{eqX}
  equação
\end{equation}
A equação \verb!(\ref{eqX})! é usada para mostrar que...
\end{verbatim}
\end{framed}

\item Evitar o uso excessivo de subseções;

\item Evitar a utilização de espaçadores 
\verb!\vspace!, \verb!\hspace!, etc.;

\item Definições, lemas, proposições, teoremas, etc. devem
ser numerados de acordo com a seção onde estão inseridos.  Há
comandos pré-definidos para sua numeração automática, são eles:
\texttt{defTEMAi}, \texttt{lemmaTEMA},  \texttt{thmTEMA} e 
\texttt{coroTEMAi},
para artigos em inglês, e  \texttt{defTEMAp}, \texttt{lemaTEMA}, 
\texttt{teoTEMA} e \texttt{coroTEMAp}, para artigos em português.

\item Para início e fim de demonstração utilize os comandos 
\verb!\begin{proof}! e \verb!\end{proof}!, respectivamente.

\item Em inglês, a palavra {\it Demonstração} será
substituída automaticamente por {\it Proof};

\begin{framed}
\begin{verbatim}
\begin{teoTEMA}[Desigualdade triangular]
\label{teoDT}
Se $a$, $b$ são números reais quaisquer, então
\[ |a+b| \leq |a| + |b| \]
\end{teoTEMA}
\begin{proof}
Coloque aqui a demonstração.
\end{proof}
\begin{coroTEMAp}
Se $a_1, a_2, \ldots, a_n$ são $n$ números reais, então
\[ |a_1+a_2+\cdots+a_n| \leq |a_1|+|a_2|+\cdots+|a_n|. \]
\end{coroTEMAp}
\end{verbatim}
\end{framed}

\begin{framed}
\begin{teoTEMA}[Desigualdade triangular]
\label{teoDT}
Se $a$, $b$ são números reais quaisquer, então
\[ |a+b| \leq |a| + |b| \]
\end{teoTEMA}
\begin{proof}
Coloque aqui a demonstração.
\end{proof}
\begin{coroTEMAp}
Se $a_1, a_2, \ldots, a_n$ são $n$ números reais, então
\[ |a_1+a_2+\cdots+a_n| \leq |a_1|+|a_2|+\cdots+|a_n|. \]
\end{coroTEMAp}
\end{framed}

\end{itemize}



%%%%%%%%%%%%%%%%%%%%%%%%%%%%%%%%%%%%%%%%%%%%
\section{Sobre referências bibliográficas}

As referências bibliográficas devem ser feitas utilizando-se 
o pacote \texttt{bibtex}. A revista TEMA adota o padrão IEEE-TR,
que ordena as referências por ordem de citação, e inclui 
uma abreviação padrão dos nomes dos autores. O autor deve 
criar um arquivo formato \texttt{.bib} e incluí-lo no texto
através dos comandos:

\begin{framed}
\begin{verbatim}
\bibliographystyle{ieeetr}
\bibliography{nome-do-arquivo}
\end{verbatim}
\end{framed}

Um exemplo de arquivo \texttt{.bib} está descrito a seguir:
\begin{framed}
\begin{verbatim}
@article{Ausas:2010,
   Author = {Ausas, R F and Sousa, Fabricio S and Buscaglia, G C},
   Journal = {Comput Methods Appl Mech Engrg},
   Number = {17-20},
   Pages = {1019-1031},
   Title = {{An improved finite element space for 
   discontinuous pressures}},
   Volume = {199},
   Year = {2010}}
@inproceedings{Silva:2014,
   Author = {Lino M. Silva and Aurelio R. L. Oliveira},
   Booktitle = {Proceeding Series of the Brazilian Society of
   Computational and Applied Mathematics},
   Pages = {1-7},
   Publisher = {SBMAC},
   Title = {Melhoria do desempenho da fatora{\c c}{\~a}o 
   controlada de {C}holesky no precondicionamento de sistemas 
   lineares oriundos dos m{\'e}todos de pontos interiores},
   Volume = {3},
   Year = {2015}}
   
@phdthesis{Linhares:1968,
   Address = {S{\~a}o Carlos, SP},
   Author = {Odelar L. Linhares},
   School = {EESC, Universidade de S\~ao Paulo},
   Title = {Sobre a racionaliza{\c c}{\~a}o de dois 
   algoritmos num{\'e}ricos},
   Year = {1968}}
@book{Leveque:1998,
   Author = {Randall J. Leveque},
   Publisher = {Cambridge},
   Title = {Finite volume methods for hyperbolic problems},
   Year = {1998}}
\end{verbatim}
\end{framed}

As citações devem ser feitas com o comando \verb!\cite{}!.
Como exemplo, as referências bibliográficas \cite{Chihara:1978,
Leveque:1998} referem-se a livros, as referências
\cite{Ausas:2010,Cordeiro:2013,Courant:1943} são artigos em
periódicos, a referência \cite{Linhares:1968} é um exemplo de 
tese de doutorado e finalmente as referências
\cite{Gautschi:1981,Jones:1986,Silva:2014}
referem-se a artigos em anais de congressos científicos.

\section*{Agradecimentos}
Aqui os autores poderão expressar seus agradecimentos a
entidades ou pessoas que ajudaram de alguma forma a realização do
trabalho. Agradecimentos a agências de fomento à pesquisa poderão
ser feitas na página inicial usando notas de rodapé \verb!\thanks{}!
ligadas ao título do trabalho ou autores. {\bf Este item é facultativo.}

\begin{abstract}
{\bf Abstract}. This document, which was prepared using the class
file \texttt{TEMA.cls}, provides some important information for the
authors who intend to submit papers to TEMA.
\end{abstract}

\bibliographystyle{ieeetr}
\bibliography{Modelo-TEMA}

\end{document}
